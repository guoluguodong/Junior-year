\section{设计简介}
本设计实现了简单的 MIPS CPU, 是一个实现 52 条基本指令和 5 条特权指令的五级流水线 CPU,实现精确异常处理,实现了 2 路组相联 LRU 替换策略的写回 cache(也实现了 4 路组相联 PLRU 替换策略的写回 cache)优化。

基于在《计算机组成与结构》课程实验四中实现的简易五级流水线 CPU,通过改写 main\_decoder 模块和 alu\_decoder 模块、扩展 datapath 和 hazard 控制模块,将原先实现的 10 指令拓展至 52 条基础指令,通过了按指令类型划分的各组测试;再添加 CP0 协处理器和实现 5 条特权指令,实现精确异常处理;并将顶层模块修改为 SRAM 接口,通过了SOC 测试的 89 个测试点;随后将 SRAM 接口转换为类 SRAM 接口,添加 cache,采用龙芯杯提供的转接桥连接 AXI 总线,通过了 AXI 功能测试的 89 个测试点和性能测试的 10 个测试点;最后生成比特流文件,在 FPGA 开发板上测试 CPU 的性能。实验观察发现,性能测试分数主要与 cache 块数呈正相关,而在 cache 块数相同时,使用组相联写回 cache 将导致最大 CPU 时钟频率降低而未见性能测试得分有明显提高;在使用块数为 1024 的直接映射写透 cache、CPU 时钟频率为 44MHZ 时,性能测试分数为 7.619。

\subsection{小组分工说明}

\begin{itemize}
    \item \stunamea:配置并测试实验环境;扩展逻辑、算术指令以外的指令和数据通路;完成 HILO 寄存器、CP0 寄存器读写逻辑;完成冒险处理逻辑;添加异常处理模块;完成对 52 条基础指令的分类功能测试、SRAM 接口的连线以及 SRAM 功能测试;添加 AXI 接口并完成 AXI 功能测试;参与部分性能测试;参与 cache 的设计和代码实现;协助代码调试工作;参与实验报告的校正。
    \item \stunameb:参与配置并测试实验环境;分析数据通路并绘制数据通路图;扩展逻辑、算术运算指令;参与 HILO 寄存器、CP0 寄存器读写逻辑实现;参与数据通路的修改;完成代码检查和调试工作;参与 cache 的设计和代码实现并绘制 cache 状态图;参与 SRAM 功能测试和 AXI 性能测试;完成实验报告的编写。
\end{itemize}
