\section{总结}

\subsection{总结设计思路}

本设计实现了简单的 MIPS CPU, 是一个实现 52 条基本指令和 5 条特权指令的五级流水线 CPU,实现精确异常处理,实现了 2 路组相联 LRU 替换策略的写回 cache(也实现了 4 路组相联 PLRU 替换策略的写回 cache)优化。

\subsection{收获}

通过硬件综合设计这门课程,收获良多。首先是回顾了《组成原理》的内容,其次学以致用,设计了 1 个简单的 cpu,能更加熟练地查阅资料学习并写出自己的东西。

\subsection{感受}

在课程中,我们遇到很多困难,感谢教材\cite{雷思磊2014自己动手写CPU,计算机组成与设计,computer-organization-and-design}提供的详细指导、龙芯中科提供的材料\cite{类SRAM接口说明,系统能力培养大赛”MIPS指令系统规范_v1.01}、实验文档\cite{重庆大学硬件综合设计实验文档}和教学视频\cite{CQU硬件综合设计录播,2022重庆大学硬件综合设计-MIPS组资源包的使用}、在 GitHub 网站上开源的往届作品\cite{A-simple-MIPS-CPU}。

\section{供同学们吐槽之用。有什么问题都可以直接写在这。}

\subsection{\stunamea}

感觉 trace 机制挺难驾驭。一个问题是在写寄存器堆的使能信号只有不到一个时钟周期为 4'hf 时,尽管程序能够正常运行,但 ref\_wb\_pc 可能不会更新而导致错误。我们设计的路线是在《计算机组成与设计》实验四的基础上扩展指令、封装 SRAM 接口、再封装 AXI 接口完成整个项目。因此我希望能够完成一份能够同时通过 SRAM 功能测试、AXI 功能测试和性能测试的代码。但是由于 trace 机制和多周期访存带来的阻塞问题,我最终没有办法实现这个目标,最终针对 SRAM 接口和 AXI 接口分别提交了两份极相似的代码。

\subsection{\stunameb}

个人感觉课程的难度在于debug上,前期debug不熟练,后期性能测试遇到问题比较深。性能分数虽然强制要求,但跑出高分确实很有意思。Cache可以结合上学期《组成原理》的内容,感觉比较有意思。感谢学长们的帮助,特别是学长们开源精神,这将被为21级的我们传承下去,帮助更多有需要的同学。
